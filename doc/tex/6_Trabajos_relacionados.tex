\capitulo{6}{Trabajos relacionados}

TODO: Revisar y dar formato, posiblemente haya errores.

Este capítulo presenta una revisión de la literatura y los trabajos existentes que son relevantes para el proyecto. El objetivo de este estado del arte es doble: por un lado, fundamentar las decisiones tomadas durante el desarrollo, mostrando que se basan en el conocimiento actual del campo; y por otro, posicionar este trabajo dentro del panorama de la investigación en detección de \textit{malware}, destacando tanto sus similitudes con otros enfoques como sus aportaciones propias. La revisión se ha estructurado en tres grandes áreas: una visión general del \textit{malware}, un análisis de los métodos de detección estática en Android y una descripción de los datasets más importantes.

% TODO: Cambiar título a algo más profesional
\section{Malware en general}

Para poder abordar un problema tan complejo como la detección de malware, la primera fase del proyecto consistió en una investigación para asentar las bases teóricas. Los trabajos presentados en esta sección sirvieron como una introducción fundamental al campo de la ciberseguridad, explicando qué es el malware, sus diferentes tipos y las metodologías generales que se emplean para su análisis.

\subsection{A review on malware analysis for IoT and Android system}

El trabajo de Yadav y Gupta (2022)~\cite{yadav2022review} ofrece una revisión exhaustiva y muy accesible del panorama del malware, con un enfoque particular en los sistemas Android e IoT. El artículo comienza introduciendo los conceptos básicos de la seguridad informática y explora las vulnerabilidades comunes que los atacantes explotan en estos entornos. Una de sus aportaciones más valiosas es la descripción detallada del "plan de explotación" que suelen seguir los atacantes, lo que ayuda a comprender la motivación y las fases de un ciberataque.

Los autores evalúan tanto el análisis estático como el dinámico, concluyendo que una estrategia de detección robusta debería, idealmente, combinar ambos enfoques. Además, discuten otras tecnologías de defensa como los Honeypots o los Sistemas de Detección de Intrusos (IDS).

Aportación al proyecto: Este artículo fue una pieza clave en la fase inicial de investigación. Proporcionó una visión de 360 grados sobre el malware y sus métodos de análisis, sirviendo como una excelente introducción para asentar los conocimientos teóricos necesarios. La clara defensa de la importancia del análisis estático reforzó la viabilidad del enfoque elegido para este proyecto.

\subsection{A Static Approach for Malware Analysis: A Guide to Analysis Tools and Techniques}

En este artículo, Nair et al. (2023)~\cite{nair2023static} se centran exclusivamente en el análisis estático, presentándolo como la primera línea de defensa contra el malware. La publicación funciona como una guía práctica que repasa las diferentes técnicas y herramientas disponibles para inspeccionar archivos sospechosos sin necesidad de ejecutarlos. Se detallan los distintos tipos de malware (troyanos, gusanos, rootkits, etc.) y se abordan los desafíos a los que se enfrentan los analistas, como las técnicas de ofuscación de código que los desarrolladores de malware utilizan para evadir la detección.

El estudio subraya la importancia de los escáneres, el análisis de cadenas de texto y el pattern matching como métodos fundamentales del análisis estático. También destaca la necesidad de que los analistas sepan cómo manejar archivos empaquetados (packed) para poder examinar su contenido real.

Este trabajo fue de gran utilidad para profundizar en la metodología central de nuestro proyecto. Mientras que otros estudios hablan del análisis estático de forma general, este artículo nos proporcionó una visión más detallada de las técnicas específicas y los problemas prácticos, como la ofuscación, que debíamos tener en cuenta a la hora de diseñar nuestro extractor de características.

\section{Análisis estático de \textit{malware} en Android}

Una vez sentadas las bases, la investigación se centró en el nicho específico de este proyecto: la aplicación de técnicas de análisis estático y, más concretamente, de inteligencia artificial, para la detección de malware en el ecosistema Android. Los trabajos de esta sección fueron cruciales para entender el estado del arte, validar la dirección del proyecto y tomar decisiones de diseño clave.

\subsection{The Android Malware Static Analysis: Techniques, Limitations, and Open Challenges}

Aunque es un trabajo de 2018, el estudio de Bakour et al.~\cite{8566573} resultó ser una fuente de motivación muy importante. El artículo realiza una revisión exhaustiva de más de 80 frameworks de análisis estático para Android, identificando sus técnicas, pero sobre todo, sus limitaciones y los desafíos que quedaban por resolver en aquel momento. Una de sus contribuciones más interesantes es la categorización de las características estáticas en cuatro grupos: basadas en el manifiesto, en el código, en la semántica y en los metadatos de la aplicación.

El estudio culmina con un caso práctico en el que se demuestra que los antivirus comerciales y las herramientas académicas de la época tenían serios problemas para detectar malware que utilizaba técnicas de ofuscación. Los autores concluyen que existía una "necesidad urgente" de herramientas más precisas y robustas.

Este paper fue fundamental para justificar la relevancia de nuestro trabajo. Confirmó que, incluso hace pocos años, el campo del análisis estático tenía carencias significativas, validando la necesidad de explorar nuevos enfoques como el nuestro. Además, su taxonomía de características sirvió de inspiración durante la fase de diseño de nuestro propio extractor.

\subsection{Droidmat: Android malware detection through manifest and api calls tracing}

El trabajo de Wu et al. (2012)~\cite{wu2012droidmat} es uno de los primeros ejemplos de un sistema que aplica aprendizaje automático al análisis estático en Android. Los autores presentan DroidMat, una herramienta que extrae características del manifiesto, los mensajes Intent y las llamadas a la API para caracterizar el comportamiento de una aplicación. Posteriormente, utiliza algoritmos de clustering (k-Means) y clasificación (k-NN) para distinguir entre aplicaciones benignas y maliciosas.

En su evaluación, los autores afirman que DroidMat no solo obtenía una tasa de recall superior a la de Androguard (en su versión de 2011), sino que también era significativamente más rápido en su análisis.

Este estudio fue una referencia muy valiosa porque demostraba, ya en 2012, que la idea de combinar la extracción de características estáticas con el aprendizaje automático era un camino viable y prometedor. Sirvió como un ejemplo concreto de qué tipo de características (permisos, API calls) eran predictivas y proporcionó un punto de partida para entender cómo se podían aplicar los modelos clásicos de ML a este problema.

\subsection{Android mobile malware detection using machine learning: A systematic review}

Este artículo de Senanayake et al. (2021)~\cite{senanayake2021android} es una revisión sistemática de la literatura que analiza cómo se ha aplicado el aprendizaje automático, y en especial el deep learning (DL), a la defensa contra el malware en Android. Tras revisar 132 estudios publicados entre 2014 y 2021, los autores concluyen que hay una tendencia clara a abandonar las reglas manuales y el ML tradicional en favor de los modelos de DL, debido a la capacidad de estos últimos para abstraer características de forma más potente y combatir las técnicas de evasión avanzadas.

El estudio no solo se centra en la detección, sino que también discute las tendencias de la investigación, los principales desafíos y las futuras líneas de trabajo en el campo de la defensa contra el malware en Android basada en DL.

Este trabajo fue crucial para validar nuestra elección tecnológica. Nos proporcionó una visión panorámica y actualizada del estado de la investigación, confirmando que el uso de redes neuronales profundas no era una idea aislada, sino la dirección principal hacia la que se movía el campo. Sirvió para contextualizar nuestro proyecto dentro de las tendencias más recientes y nos dio confianza en el enfoque seleccionado.

\subsection{A Method for Automatic Android Malware Detection Based on Static Analysis and Deep Learning}

El trabajo de İbrahim et al. (2022)~\cite{9936621} ha sido la principal fuente de inspiración y el punto de partida fundamental para este proyecto. El artículo aborda exactamente el mismo problema que nosotros: la creación de un sistema automático de detección de malware en Android basado exclusivamente en análisis estático y un modelo de deep learning. Los autores proponen un método que consiste en recolectar un gran número de características estáticas, incluyendo dos propuestas por ellos mismos, para luego alimentar un modelo de red neuronal funcional construido con la API de Keras.

Para su evaluación, crearon un dataset propio con más de 14.000 muestras y realizaron dos experimentos: uno de clasificación binaria (malware vs. benigno) y otro de clasificación multiclase (diferenciando entre familias de malware). Sus resultados fueron extraordinariamente prometedores, alcanzando un F1-Score del 99.5\% en la detección binaria, superando a otros trabajos relacionados.

Este paper fue la raíz de nuestro proyecto. Sirvió como el "plano" inicial, demostrando que era posible alcanzar una precisión altísima utilizando únicamente características estáticas. La metodología que describen, incluyendo el uso de Androguard para la extracción de características, fue la base sobre la que empezamos a construir nuestro propio prototipo. Sin embargo, el artículo omite ciertos detalles cruciales de implementación, como la arquitectura exacta de la red neuronal o cómo se entrenaron los modelos clásicos con los que se comparan. Esta falta de detalle nos motivó a no solo replicar su idea, sino a profundizar en estos aspectos, realizando nuestra propia optimización de hiperparámetros y un análisis comparativo más transparente y detallado.

\subsection{MAPAS: a practical deep learning-based android malware detection system}

Kim et al. (2022)~\cite{kim2022mapas} proponen en su trabajo MAPAS, un sistema de detección de malware con un enfoque muy interesante y orientado a la eficiencia. Al igual que nosotros, utilizan el deep learning, en su caso una Red Neuronal Convolucional (CNN), para analizar características estáticas, concretamente grafos de llamadas a la API. Sin embargo, la gran diferencia es que no utilizan la CNN como el clasificador final. En su lugar, la emplean únicamente como un potente extractor de características para descubrir patrones comunes en el malware.

La clasificación final la realiza un algoritmo mucho más ligero, que simplemente calcula la similitud entre los patrones de una nueva aplicación y los patrones de malware ya conocidos. Gracias a este diseño, afirman que su sistema es mucho más rápido y consume hasta diez veces menos memoria que otras aproximaciones, lo que lo haría viable para ser ejecutado directamente en un dispositivo móvil.

Aportación al proyecto: Este trabajo nos ofreció una perspectiva alternativa muy valiosa sobre cómo utilizar el deep learning. Mientras nuestro proyecto utiliza el embedder de la red neuronal para potenciar a modelos clásicos, MAPAS utiliza una CNN para alimentar a un clasificador basado en similitud. Esta filosofía de usar el DL para el aprendizaje de representaciones y delegar la clasificación final a un modelo más simple es una idea muy poderosa que resuena con nuestras propias conclusiones, reforzando la idea de que los enfoques híbridos son un camino muy prometedor.

\section{\textit{Datasets} de \textit{malware} para Android}

Ningún proyecto de aprendizaje automático puede existir sin datos. Esta sección final revisa los trabajos que introducen los datasets más relevantes del campo o las herramientas que facilitan el acceso a grandes volúmenes de aplicaciones, recursos que han sido indispensables para el desarrollo de este trabajo.

\subsection{Drebin: Effective and explainable detection of android malware in your pocket}
El trabajo de Arp et al. (2014)~\cite{arp2014drebin} no solo propone un método de detección, sino que también introduce el que se convertiría en uno de los datasets más icónicos y utilizados en la investigación de malware en Android: el dataset Drebin. Este conjunto de datos contiene 5.560 muestras de malware y más de 123.000 aplicaciones benignas, junto con un conjunto de características estáticas extraídas de cada una de ellas, como permisos, llamadas a API o componentes del manifiesto.

El objetivo original de Drebin era crear un sistema lo suficientemente ligero como para poder ejecutarse directamente en un teléfono. Aunque el artículo no detalla con precisión el proceso exacto para extraer las características de una nueva aplicación, el dataset que liberaron ha sido una referencia para la comunidad durante años.

Aportación al proyecto: El dataset Drebin fue el recurso fundamental sobre el que se construyó y validó el prototipo inicial de este proyecto. Nos permitió realizar las primeras pruebas de concepto de forma rápida, sin tener que pasar por la compleja fase de recolección y procesamiento de datos, lo que fue clave para confirmar la viabilidad de nuestro enfoque antes de embarcarnos en la creación de nuestro propio dataset.

\subsection{Dynamic android malware category classification using semi-supervised deep learning}

En este trabajo, Mahdavifar et al. (2020)~\cite{mahdavifar2020dynamic} presentan un nuevo dataset llamado CICMalDroid2020. Este conjunto de datos es particularmente interesante porque incluye más de 17.000 muestras recientes y, a diferencia de Drebin, contiene tanto características estáticas como dinámicas (obtenidas de la ejecución de las aplicaciones). Además, el artículo propone un enfoque de aprendizaje semisupervisado para clasificar el malware en diferentes categorías (Adware, Banking, etc.), una técnica muy útil cuando se dispone de pocos datos etiquetados.

Aportación al proyecto: Aunque nuestro proyecto se centra exclusivamente en el análisis estático, este paper fue una referencia importante para entender cómo son y cómo se construyen los datasets modernos. Nos proporcionó una visión actualizada de los tipos de características que se consideran relevantes hoy en día y nos sirvió como punto de comparación para el diseño de nuestro propio proceso de recolección de datos.

\subsection{Deep ground truth analysis of current android malware}
El trabajo de Wei et al. (2017)~\cite{wei2017deep} tiene un objetivo distinto a los demás. En lugar de crear un dataset para entrenar modelos, su meta fue crear un conjunto de datos de "verdad fundamental" (ground truth) a través de un análisis manual exhaustivo y profundo. Analizaron casi 25.000 muestras de malware y las clasificaron manualmente en 71 familias y 135 variedades distintas, documentando en detalle el comportamiento específico de cada una.

El resultado es un recurso de un valor incalculable para entender el ecosistema del malware en Android, no desde un punto de vista estadístico, sino cualitativo.

Aportación al proyecto: Este estudio fue una fuente de conocimiento crucial para comprender el contexto del problema. Nos ayudó a entender la diversidad y la complejidad del malware que pretendíamos detectar. Tener una idea clara de las diferentes familias y sus tácticas fue muy útil para interpretar los resultados de nuestro modelo y entender por qué algunas muestras podían ser más difíciles de clasificar que otras.

\subsection{AndroZoo: Collecting Millions of Android Apps for the Research Community}
Allix et al. (2016)~\cite{Allix:2016:ACM:2901739.2903508} presentan AndroZoo, que no es un dataset en sí mismo, sino un inmenso repositorio de aplicaciones Android. Se trata de un proyecto en continuo crecimiento que recolecta millones de archivos APK de diversas fuentes, incluyendo la tienda oficial de Google Play, mercados de terceros y colecciones de malware como VirusShare. El proyecto pone toda esta colección a disposición de la comunidad investigadora a través de una API.

Para cada aplicación, AndroZoo también proporciona los resultados de su análisis por parte de decenas de antivirus comerciales, lo que permite tener una etiqueta de "malicioso" o "benigno" con un cierto grado de confianza.

Aportación al proyecto: AndroZoo fue, sin duda, la herramienta más indispensable para la fase de construcción de nuestro dataset. Todo el conjunto de datos propio sobre el que se ha entrenado y evaluado el modelo final de este proyecto se ha creado a partir de miles de muestras, tanto benignas como maliciosas, descargadas directamente del repositorio de AndroZoo. Sin el acceso a esta increíble colección, la creación de nuestro dataset habría sido prácticamente imposible.
