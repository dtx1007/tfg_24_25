\capitulo{6}{Trabajos relacionados}

Este capítulo presenta una revisión de la literatura y los trabajos existentes que son relevantes para el proyecto. El objetivo de este estado del arte es doble: por un lado, fundamentar las decisiones tomadas durante el desarrollo, mostrando que se basan en estudios y técnicas usadas actualmente en el campo; y por otro, posicionar este trabajo dentro del panorama de la investigación en detección de \textit{malware}, destacando tanto sus similitudes con otros enfoques como sus aportaciones propias. La revisión se ha estructurado en tres áreas: una visión general acerca del \textit{malware}, un análisis de otras investigaciones y papers que han contemplado métodos de detección estática de \textit{malware} Android y un último apartado hablando acerca de algunos \textit{datasets} bastante comunes en este ámbito.

\section{Primer contacto con el \textit{malware}}

Para poder abordar un problema tan complejo como la detección de \textit{malware}, la primera fase del proyecto consistió en una investigación para asentar una buena base teórica acerca del mundillo del \textit{malware} y todo el panorama que este cubre. Los trabajos presentados en esta sección sirvieron como una introducción al campo de la ciberseguridad, explicando qué es el \textit{malware}, sus diferentes tipos y las técnicas que se emplean para su análisis.

\subsection{A review on malware analysis for IoT and Android system}

El trabajo de Yadav y Gupta (2022)~\cite{yadav2022review} ofrece una revisión exhaustiva y muy accesible del panorama del \textit{malware}, con un enfoque particular en los sistemas que usan Android y los dispositivos IoT. El artículo comienza introduciendo los conceptos básicos de la seguridad informática y explora las vulnerabilidades comunes que los atacantes explotan en estos entornos. Una de sus aportaciones más valiosas es la descripción detallada del <<plan de explotación>> que suelen seguir los atacantes, lo que ayuda a comprender la motivación y las fases de un ciberataque.

Los autores evalúan tanto el análisis estático como el dinámico, concluyendo que una estrategia de detección óptima debería, idealmente, combinar ambos enfoques. Además, discuten otras tecnologías de defensa como los Honeypots o los Sistemas de Detección de Intrusos (IDS).

Este artículo fue uno de los más importantes durante la fase inicial de investigación. Puesto que, proporcionó una visión amplia y general sobre el \textit{malware} y sus métodos de análisis, sirviendo como una excelente introducción al tema.

\subsection{A Static Approach for Malware Analysis: A Guide to Analysis Tools and Techniques}

En este artículo, Nair et al. (2023)~\cite{nair2023static} se centran exclusivamente en el análisis estático, presentándolo como la primera línea de defensa contra el \textit{malware}. La publicación funciona como una guía práctica que repasa las diferentes técnicas y herramientas disponibles para inspeccionar archivos sospechosos sin necesidad de ejecutarlos. Se detallan los distintos tipos de \textit{malware} (troyanos, gusanos, \textit{rootkits}, etc.) y abordan algunos de los desafíos y problemas a los que se enfrentan hoy en día los analizadores de \textit{malware}, como por ejemplo, las técnicas de ofuscación de código, las cuales permiten <<cifrar>> el código de un programa de tal manera que dificulte su análisis.

El estudio subraya la importancia de los analizadores, el análisis de cadenas de texto y el \textit{pattern matching} como métodos fundamentales del análisis estático. También destaca la necesidad de que los analistas sepan cómo manejar archivos empaquetados (\textit{packed}) para poder examinar su contenido real, esto sería equiparable a cómo funcionan las APK en Android.

Este trabajo fue de gran utilidad para profundizar en la metodología central del proyecto. Mientras que otros estudios hablan del análisis estático de forma general, este artículo proporciona una visión más detallada de las técnicas específicas y los problemas prácticos, como la ofuscación, que se han de tener en cuenta a la hora de diseñar un sistema que sea capaz de extraer características de archivos maliciosos.

\section{Análisis estático de \textit{malware} en Android}

Una vez sentadas las bases, la investigación se centró en el nicho específico de este proyecto: la aplicación de técnicas de análisis estático y, más concretamente, de inteligencia artificial, para la detección de \textit{malware} en el ecosistema de Android. Los trabajos de esta sección fueron los que más influenciaron la dirección que se ha tomado en el trabajo y son principalmente la razón por la cual se siguió adelante con el mismo.

\subsection{A Method for Automatic Android Malware Detection Based on Static Analysis and Deep Learning}

El trabajo de İbrahim et al. (2022)~\cite{9936621} ha sido la principal fuente de inspiración y el punto de partida de este proyecto. El artículo aborda exactamente el mismo problema que el que se intenta resolver en este trabajo: la creación de modelo de aprendizaje automático capaz de detectar \textit{malware} en Android basado exclusivamente en el análisis estático de dichos archivos. Los autores proponen un método que consiste en recolectar un gran número de características estáticas, incluyendo dos propuestas por ellos mismos, para luego alimentar un modelo de red neuronal construido con la API de Keras.

Para su evaluación, crearon un \textit{dataset} propio con más de 14\,000 muestras y realizaron dos experimentos: uno de clasificación binaria (\textit{malware} vs. benigno) y otro de clasificación multiclase (diferenciando entre distintas familias de \textit{malware}). Sus resultados fueron muy prometedores, alcanzando un \textit{$F_1$-Score} del 99,5\% en la detección binaria, superando a otros trabajos relacionados.

Este \textit{paper} fue la raíz del proyecto. Sirvió para demostrar que era posible alcanzar una precisión alta utilizando únicamente características estáticas. La metodología que describen, incluyendo el uso de Androguard para la extracción de características, fue la base sobre la que se empezó a construir el prototipo planteado. Sin embargo, hay que destacar que, a pesar de ser un muy buen \textit{paper}, el artículo omite ciertos detalles cruciales de implementación, como la arquitectura exacta de la red neuronal o cómo se entrenaron los modelos clásicos con los que se comparan. Esta falta de detalle motivó no solo a replicar su idea, sino a profundizar en estos aspectos, realizando pruebas propias de optimización de hiperparámetros y un análisis comparativo más transparente y detallado. Además de realizar un análisis de interpretabilidad a los diferentes modelos para aumentar la confianza de estos.

\subsection{MAPAS: a practical deep learning-based android malware detection system}

Kim et al. (2022)~\cite{kim2022mapas} proponen en su trabajo MAPAS, un sistema de detección de \textit{malware} con un enfoque muy interesante y orientado a la eficiencia. Al igual que en este caso, utilizan un modelo de \textit{deep learning}, pero, con la diferencia de que en su caso usan una Red Neuronal Convolucional (CNN), para analizar características estáticas, concretamente grafos de llamadas a la API. Sin embargo, la gran diferencia es que no utilizan la CNN como el clasificador final. En su lugar, la emplean únicamente como un extractor de características para descubrir patrones comunes en el \textit{malware}.

La clasificación final la realiza un algoritmo mucho más ligero, que simplemente calcula la similitud entre los patrones de una nueva aplicación y los patrones de \textit{malware} obtenidos mediante la CNN. Gracias a este diseño, afirman que su sistema es mucho más rápido y consume hasta diez veces menos memoria que otras aproximaciones, lo que lo haría viable para ser ejecutado directamente en un dispositivo móvil.

Este trabajo ofreció una perspectiva alternativa muy valiosa sobre otra línea posible de investigación del proyecto. Mientras que en este caso se emplea el \textit{embedder} de la red neuronal para potenciar a modelos clásicos, MAPAS utiliza una CNN para alimentar a un clasificador basado en similitud. Esta filosofía de usar el \textit{deep learning} para el aprendizaje de representaciones y delegar la clasificación final a un modelo más simple es una idea muy poderosa que resuena con las conclusiones propias, reforzando la idea de que una solución mixta parece ser la mejor forma de atacar el problema, sobretodo en el caso en el que se deseara llevar este tipo de analizadores a dispositivos móviles donde la cantidad de recursos disponible es mucho más limitada.

\subsection{Android mobile malware detection using machine learning: A systematic review}

Este artículo de Senanayake et al. (2021)~\cite{senanayake2021android} es una revisión sistemática de la literatura que analiza cómo se ha aplicado el aprendizaje automático, y en especial el \textit{deep learning} (DL), a la defensa contra el \textit{malware} en Android. Tras revisar 132 estudios publicados entre 2014 y 2021, los autores concluyen que hay una tendencia clara a abandonar las reglas manuales y el ML tradicional en favor de los modelos de DL, debido a la capacidad de estos últimos para abstraer características de forma más potente, siendo capaces de combatir mejor contra técnicas de evasión modernas.

El estudio no solo se centra en la detección, sino que también discute las tendencias de la investigación, los principales desafíos y las futuras líneas de trabajo en el campo de la defensa contra el \textit{malware} en Android basada en DL.

\subsection{The Android Malware Static Analysis: Techniques, Limitations, and Open Challenges}

Aunque es un trabajo de 2018, el estudio de Bakour et al.~\cite{8566573} resultó ser una fuente de motivación muy importante. El artículo realiza una revisión exhaustiva de más de 80 \textit{frameworks} de análisis estático para Android, identificando sus técnicas, pero sobre todo, sus limitaciones y los desafíos que quedaban por resolver en aquel momento. Una de sus contribuciones más interesantes es la categorización de las características estáticas en cuatro grupos: basadas en el manifiesto, en el código, en la semántica y en los metadatos de la aplicación.

El estudio concluye con un caso práctico en el que se demuestra que los antivirus comerciales y las herramientas académicas de la época tenían serios problemas para detectar \textit{malware} que utilizaba técnicas de ofuscación. Los autores concluyen que existía una <<necesidad urgente>> de herramientas más precisas y robustas.

Este \textit{paper} fue bastante útil como motivación para la realización del proyecto pues que, se confirma que, incluso hace pocos años, el campo del análisis estático tenía carencias significativas, validando la necesidad de explorar nuevos enfoques como el presentado en este caso.

\subsection{Droidmat: Android malware detection through manifest and api calls tracing}

El trabajo de Wu et al. (2012)~\cite{wu2012droidmat} es uno de los primeros ejemplos de un sistema que aplica aprendizaje automático al análisis estático en Android. Los autores presentan DroidMat, una herramienta que extrae características del manifiesto, los mensajes Intent y las llamadas a la API para caracterizar el comportamiento de una aplicación. Posteriormente, utiliza algoritmos de clustering ($k$-Means) y clasificación ($k$-NN) para distinguir entre aplicaciones benignas y maliciosas.

En su evaluación, los autores afirman que DroidMat no solo obtenía una tasa de \textit{recall} superior a la de Androguard (en su versión de 2011), sino que también era significativamente más rápido en su análisis.

Este estudio fue una referencia muy valiosa porque demostraba, ya en 2012, que la idea de combinar la extracción de características estáticas con el aprendizaje automático era un camino viable y prometedor.

\section{\textit{Datasets} de \textit{malware} para Android}

Al igual que todo proyecto de aprendizaje automático, es necesario tener datos de calidad para que estos salgan adelante. Esta sección final revisa los trabajos que crean algunos de los \textit{datasets} más relevantes del campo y las herramientas que se usan para llevar esto a cabo.

\subsection{Drebin: Effective and explainable detection of android malware in your pocket}
El trabajo de Arp et al. (2014)~\cite{arp2014drebin} no solo propone un método de detección, sino que también introduce uno de los \textit{datasets} más utilizados en la investigación de \textit{malware} en Android: el \textit{dataset} Drebin. Este conjunto de datos contiene 5.560 muestras de \textit{malware} y más de 123.000 aplicaciones benignas, junto con un conjunto de características estáticas extraídas de cada una de ellas, como permisos, llamadas a API o componentes del manifiesto.

El objetivo original de Drebin era crear un sistema lo suficientemente ligero como para poder ejecutarse directamente en un teléfono. Aunque el artículo no detalla con precisión el proceso exacto para extraer las características de una nueva aplicación, el \textit{dataset} que liberaron ha sido una referencia para la comunidad durante bastante tiempo.

El \textit{dataset} Drebin fue de gran ayuda en este proyecto debido a que su conjunto de características era bastante similar al del \textit{paper} que se utilizó de referencia. Este detalle junto con el hecho de que el \textit{dataset} está fuertemente desbalanceado en cuanto a la clase negativa fueron dos factores muy útiles que permitieron empezar el desarrollo temprano de un prototipo bastante bueno que asentarías las bases del modelo final que se desarrolla a lo largo del trabajo. Cabe destacar que, dado la falta de información acerca de su proceso de creación, también inspiró a la creación de un \textit{dataset} propio.

\subsection{Dynamic android malware category classification using semi-supervised deep learning}

En este trabajo, Mahdavifar et al. (2020)~\cite{mahdavifar2020dynamic} presentan un nuevo \textit{dataset} llamado CICMalDroid2020. Este conjunto de datos es bastante interesante porque incluye más de 17.000 muestras recientes y, a diferencia de Drebin, contiene tanto características estáticas como dinámicas (obtenidas de la ejecución de las aplicaciones). Además, el artículo propone un enfoque de aprendizaje semisupervisado para clasificar el \textit{malware} en diferentes categorías (Adware, Banking, etc.), una técnica muy útil cuando se dispone de pocos datos etiquetados.

\subsection{Deep ground truth analysis of current android malware}
El trabajo de Wei et al. (2017)~\cite{wei2017deep} tiene un objetivo distinto a los demás. En lugar de crear un \textit{dataset} para entrenar modelos, su meta fue crear un conjunto de datos de <<verdad fundamental>> (\textit{ground truth}) a través de un análisis manual exhaustivo y profundo. Analizaron casi 25.000 muestras de \textit{malware} y las clasificaron manualmente en 71 familias y 135 variedades distintas, documentando en detalle el comportamiento específico de cada una.

El resultado es un recurso de un valor incalculable para entender el ecosistema del \textit{malware} en Android, no desde un punto de vista estadístico, sino cualitativo.

\subsection{AndroZoo: Collecting Millions of Android Apps for the Research Community}
Allix et al. (2016)~\cite{Allix:2016:ACM:2901739.2903508} presentan AndroZoo, que no es un \textit{dataset} en sí mismo, sino un inmenso repositorio de aplicaciones para Android. Se trata de un proyecto en continuo desarrollo que recolecta miles de archivos APK de diversas fuentes, incluyendo la tienda oficial de Google Play, mercados de terceros y colecciones de \textit{malware} como VirusShare. El proyecto pone toda esta colección a disposición de la comunidad investigadora a través de una API propia.

Para cada aplicación, AndroZoo también proporciona los resultados de su análisis por parte de decenas de antivirus comerciales, mediante el uso de VirusTotal lo que permite que uno pueda clasificar cada APK con una etiqueta de <<malicioso>> o <<benigno>> con un cierto grado de confianza.

AndroZoo fue, sin duda, una de las herramientas más importantes a la hora de construir el \textit{dataset} propio de este trabajo. Todo el conjunto de datos sobre el que se ha entrenado y evaluado el modelo final de este proyecto se ha creado a partir de miles de muestras, tanto benignas como maliciosas, descargadas directamente del repositorio de AndroZoo. Sin el acceso a esta increíble colección, la creación de dicho \textit{dataset} habría sido muchísimo más difícil sino prácticamente imposible.
