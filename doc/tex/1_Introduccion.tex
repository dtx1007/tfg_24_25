\capitulo{1}{Introducción}

La hegemonía de Android en el panorama móvil, superando el 70\% del mercado mundial~\cite{statcounterAndroid2025}, ha traído consigo una consecuencia inevitable: es el blanco predilecto del software malicioso. Para hacer frente a esta amenaza constante, este trabajo explora el potencial de la inteligencia artificial, y en concreto de las redes neuronales profundas, para automatizar la detección de \textit{malware} a través del análisis estático, una metodología segura que evita la ejecución de código potencialmente peligroso.

El recorrido de este proyecto se puede dividir en dos actos. El primero fue un acto de exploración, donde se construyó un prototipo inicial en base a un \textit{dataset} público llamado Drebin. Su éxito confirmó que una red neuronal podía aprender a diferenciar aplicaciones seguras de maliciosas, pero también nos enfrentó a una dura realidad: sin un proceso de extracción de datos transparente, el modelo estaba <<atado>> al \textit{dataset} con el que fue creado, sin capacidad para analizar nuevas amenazas. Esta limitación, lejos de ser un fracaso, fue el punto de inflexión que dio sentido y propósito a la segunda parte del viaje.

El segundo acto, fue por tanto, un ejercicio de ingeniería. Se construyó un \textit{dataset} propio de 20\,000 aplicaciones con un \textit{pipeline} de extracción de características totalmente replicable. Sobre esta nueva base, se entrenó un modelo de red neuronal que no solo alcanzó un rendimiento excelente (con un \textit{recall} cercano al 99\%), sino que también nos llevó a la conclusión más importante del trabajo: la verdadera fortaleza del modelo no residía en su clasificación final, sino en su \textit{embedder}. Este componente demostró ser una herramienta tan potente que, al usarlo para alimentar a modelos clásicos como XGBoost, estos lograron un rendimiento igual o incluso superior, validando los enfoques híbridos como una estrategia sumamente eficaz.