\capitulo{1}{Introducción}


% TODO: Introducción al documento

% Visión general del problema, Android es una plataforma mundialmente usada que compone la gran mayoría del mercado de los smartphones hoy en día. (añadir cifra y cita)

% Explicación del proyecto y lo que se ha realizado, las distitnas fases de este.

% Breve explicación de los resultados

Intro

\section{Estructura de la memoria}

El trabajo se divide en dos partes principales, una memoria (este documento) con la explicación general de lo realizado y, un documento adicional de anexos que contiene diferentes apéndices relativos al proyecto que indagan en cuestiones más específicas acerca del mismo. La estructura de ambos documentos puede verse reflejada a continuación:

\subsection{Memoria}

La memoria (este documento) sigue la siguiente estructura:

\begin{enumerate}
	\item \textbf{Introducción:} Presenta el proyecto, su contexto y motivación, y describe la estructura del documento.
	
	\item \textbf{Objetivos del proyecto:} Detalla las metas específicas que el proyecto busca alcanzar, tanto a nivel de funcionalidades como de requisitos técnicos.
	
	\item \textbf{Conceptos teóricos:} Establece el marco teórico necesario para poder comprender el documento.
	
	\item \textbf{Técnicas y herramientas:} Describe el conjunto de librerías y otras tecnologías que se han empleado para desarrollar el proyecto.
	
	\item \textbf{Aspectos relevantes del desarrollo del proyecto:} Narra el proceso realizado a lo largo del trabajo, explicando las fases, las decisiones tomadas y cómo se llegó a la solución.
	
	\item \textbf{Trabajos relacionados:} Análisis del estado del arte relacionado con el proyecto.
	
	\item \textbf{Conclusiones y líneas de trabajo futuras:} Reflexiones e ideas finales acerca del proyecto y aspectos a mejorar en un futuro.
\end{enumerate}

\subsection{Anexos}

El documento de anexos tiene los siguientes apéndices:

\begin{enumerate}[label=\Alph*]
	\item \textbf{Plan de proyecto software:} Contiene la planificación del proyecto y los estudios de viabilidad de este.
	 
	\item \textbf{Especificación de requisitos:} Expone los requisitos funcionales y no funcionales del sistema, catálogo de requisitos y especificaciones adicionales.
	
	\item \textbf{Especificación de diseño:} Describe el diseño de datos y arquitectónico del proyecto.
	
	\item \textbf{Documentación técnica de programación:} Guía destinada a los desarrolladores que explica la estructura del código, las decisiones de implementación y los pasos para configurar el entorno de desarrollo.
	
	\item \textbf{Documentación de usuario:} Manual de instrucciones dirigido al usuario final, que explica cómo utilizar la aplicación y sus funcionalidades.
\end{enumerate}

\subsection{Recursos adicionales}

Además de la memoria y los anexos, se proporciona el enlace a la aplicación creada y al repositorio del proyecto.

\begin{enumerate}
	\item \textbf{Aplicación web:} 
	
	\item \textbf{Repositorio del proyecto:} 
\end{enumerate}
