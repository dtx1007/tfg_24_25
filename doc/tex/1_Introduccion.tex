\capitulo{1}{Introducción}

El sistema operativo Android se ha consolidado como la plataforma móvil dominante a nivel mundial, ostentando una cuota de mercado que supera el 70\% en los últimos años~\cite{statcounterAndroid2025}. Esta hegemonía lo convierte, inevitablemente, en el principal objetivo para los desarrolladores de \textit{software} malicioso o \textit{malware}. La escala y la velocidad con la que aparecen nuevas amenazas hacen que los métodos de detección manuales sean insuficientes, creando una necesidad urgente de sistemas automatizados, eficientes y, sobretodo, seguros. En este contexto, el análisis estático se presenta como un enfoque ideal, ya que permite inspeccionar la estructura y el código de una aplicación en busca de indicadores maliciosos sin la necesidad siquiera de ejecutarla, eliminando así el riesgo de infección durante el análisis.

Este trabajo documenta el desarrollo de una herramienta de estas características, que emplea técnicas de inteligencia artificial, y concretamente redes neuronales profundas, para la clasificación de aplicaciones de Android. El proyecto se dividió en dos grandes fases. La primera fue de carácter exploratorio, donde se construyó un prototipo utilizando un \textit{dataset} público (Drebin) para validar la viabilidad del enfoque. Aunque esta prueba de concepto fue exitosa, también reveló una limitación crítica y muy común en el ámbito académico: la imposibilidad de aplicar el modelo a nuevas muestras del mundo real debido a la naturaleza de <<caja negra>> del proceso de creación de dicho \textit{dataset}.

Este desafío motivó la segunda y principal fase del proyecto, que se centró en crear una solución práctica para resolver el problema. Se procedió a la creación de un \textit{dataset} propio, compuesto por 20\,000 aplicaciones, y al desarrollo de un \textit{pipeline} de extracción de características completamente transparente y replicable. Sobre esta base sólida se diseñó, entrenó y optimizó un modelo de red neuronal que no solo alcanzó un rendimiento excelente, con un \textit{recall} y un AUC (Área Bajo la Curva) cercanos al 99\%, sino que también arrojó una conclusión fundamental. Al comparar su rendimiento con el de modelos de aprendizaje automático más clásicos, se descubrió que la verdadera fortaleza de la red neuronal residía en su \textit{embedder}: el componente capaz de transformar los datos brutos en una representación numérica de alta calidad. Este \textit{embedder} permitió que modelos más simples y ligeros, como XGBoost, alcanzaran un nivel de precisión similar o incluso superior. Este hallazgo subraya que el valor de una arquitectura de \textit{deep learning} puede aportaar más que simplemente ser un clasificador de principio a fin, convirtiéndose en una poderosa herramienta capaz de habilitar todo un ecosistema de modelos eficientes. A lo largo de este documento, se detallará cada paso de este viaje, desde la investigación inicial hasta el análisis de los resultados y la creación de una aplicación web para su demostración.
