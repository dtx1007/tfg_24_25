\capitulo{4}{Técnicas y herramientas}

En este apartado se comentan y analizan las diferentes herramientas usadas en la realización de este proyecto.

\section{PyTorch vs Keras}

% TODO: Breve definición y cambiar para que use las funciones de la plantilla

\begin{table}[!ht]
	\raggedleft
	\begin{tabular}{|l|l|l|}
		\hline
		\textbf{Características} & \textbf{PyTorch} & \textbf{Keras} \\ \hline
		\textbf{Flexibilidad} & Alta, permite crear redes neuronales personalizadas y complejas. & Menos flexible, enfocado en redes estándar. \\ \hline
		\textbf{Facilidad de uso} & Requiere más código y tiene una curva de aprendizaje más pronunciada. & Fácil de usar, ideal para desarrolladores principiantes. \\ \hline
		\textbf{Personalización} & Excelente para redes personalizadas y modelos avanzados. & Limitada, más orientada a redes convencionales. \\ \hline
		\textbf{Comunidad y soporte} & Muy popular en la investigación académica y proyectos avanzados. & Amplio uso en la industria por su simplicidad. \\ \hline
		\textbf{Uso principal} & Investigación, redes neuronales complejas. & Desarrollo rápido de modelos estándar. \\ \hline
	\end{tabular}
	\caption{Comparativa entre PyTorch y Keras}
\end{table}

%\section{Vue (\textit{Framework} web)}

Esta parte de la memoria tiene como objetivo presentar las técnicas metodológicas y las herramientas de desarrollo que se han utilizado para llevar a cabo el proyecto. Si se han estudiado diferentes alternativas de metodologías, herramientas, bibliotecas se puede hacer un resumen de los aspectos más destacados de cada alternativa, incluyendo comparativas entre las distintas opciones y una justificación de las elecciones realizadas. 
No se pretende que este apartado se convierta en un capítulo de un libro dedicado a cada una de las alternativas, sino comentar los aspectos más destacados de cada opción, con un repaso somero a los fundamentos esenciales y referencias bibliográficas para que el lector pueda ampliar su conocimiento sobre el tema.


