\capitulo{3}{Conceptos teóricos}

En esta sección se definen aquellos conceptos que son necesarios conocer para comprender el resto del documento.

TODO: malware, tipos de malware más comunes, métodos de ocultación, definición de alguna herramienta usada...

\subsection{Análisis estático}
Técnica de detección de \textit{malware} que se realiza sin la necesidad de ejecutar el programa en cuestión. Este método se basa en la obtención, inspección y evaluación de las características que se pueden extraer de un archivo binario, tales como su estructura, código fuente (si está disponible), indicios de obfuscación u otras técnicas de ocultación, cadenas de texto incrustadas en este, firmas digitales, huella digital \textit{hash o signature} del archivo, secuencias de bytes concretas, cabeceras del programa, metadatos incrustados, desensamblado del ejecutable y otras propiedades que pueden ser extraídas directamente del archivo. Las ventajas que este enfoque presenta son, su simplicidad, rapidez y bajo coste computacional, ya que no requiere de entornos de ejecución específicos ni de hardware especializado para probar el comportamineto del programa. Sin embargo, el mayor problema de este tipo de análisis es su dificultad para detectar malware que utiliza técnicas avanzadas de ofuscación o cifrado, ya que estas prácticas dificultan la extracción de información útil del binario.

\subsection{Análisis dinámico}
Técnica de detección de \textit{malware} que consiste en evaluar el comportamiento de un programa mediante su ejecución en un entorno controlado, con el objetivo de observar sus interacciones con el sistema operativo, los recursos de este y otros programas. En este enfoque, se monitorean actividades como la modificación de archivos, el tráfico de red generado, la creación de procesos o la inyección de código en estos, lo cual permite identificar patrones de comportamiento asociados con programas maliciosos. A diferencia del análisis estático, el análisis dinámico ofrece una mayor precisión, ya que puede detectar comportamientos maliciosos que no son evidentes simplemente escanenado el archivo de manera estática, como el uso de técnicas de ofuscación. Sin embargo, este tipo de análisis sigue teniendo sus incovenientes, por un lado, es más complejo, requiere de más recursos computacionales y es más costoso de implementar, dado que involucra la ejecución real del código en un entorno controlado, generalmente una máquina virtual (\textit{sandbox}). Por otro lado, también es poco eficiente contra casos en los que el \textit{malware} detecta el hecho de que está siendo analizado y oculta su comportamiento malicioso. Además, puede no ser adecuado para dispositivos con recursos limitados, como dispositivos IoT o móviles, debido a sus altos requerimientos de hardware y tiempo.

\subsection{Análisis híbrido}
Metodo de detección de \textit{malware} el cual combina las fortalezas tanto del análisis estático como del dinámico. En este método, el programa se ejecuta en un entorno controlado, y durante su ejecución, se realizan \textit{dumps} de memoria de manera periódica o en respuesta a comportamientos sospechosos. Estos volcados de memoria son posteriormente analizados utilizando técnicas de análisis estático para identificar posibles  patrones maliciosos, tales como la inyección de código en procesos ajenos, manipulación de memoria que no le pertenece al programa o modificaciones en partes protegidas de la memoria pertenecientes al sistema oeprativo. Este enfoque permite una detección más precisa de \textit{malware} que utiliza técnicas avanzadas de ocultamiento, ya que combina la observación del comportamiento en tiempo real con la inspección detallada del estado de la memoria. Sin embargo, el análisis híbrido es el más complejo y costoso de implementar, ya que requiere tanto de infraestructura de virtualización como de herramientas para realizar un buen análisis de memoria. A pesar de todo, suele ofrecer los mejores resultados en términos de detección.

\subsection{Huella digital (\textit{fingerprinting})}
El fingerprinting o, la generación de huellas digitales de archivos, es una técnica utilizada para identificar de manera única un archivo mediante la aplicación de funciones criptográficas de \textit{hashing}. Este proceso consiste en calcular un \textit{hash} a partir del contenido completo del archivo utilizando algoritmos como MD5, SHA-1, SHA-256 u otros. El resultado es una cadena de longitud fija que actúa como un identificador único para ese archivo. Cualquier modificación, por mínima que sea, en el contenido del archivo resultará en un \textit{hash} completamente diferente, lo que permite detectar alteraciones o corrupciones en estos.

Esta técnica es muy utilizada en la verificación de la integridad de archivos, la detección de duplicados, y la identificación de malware conocido al comparar el \textit{hash} que este genera con una base de datos de muestras previamente catalogadas. Sin embargo, una limitación importante del \textit{fingerprinting} es su sensibilidad extrema a cambios mínimos, lo que dificulta la identificación de archivos que han sido ligeramente modificados pero que conservan una estructura o funcionalidad. Esto implica que incluso cambiar un bit en el \textit{padding} del archivo, hace que este ya no se detecte como malware al tener una huella digital diferente.

\subsection{Huella digital difusa (\textit{fuzzy hashing})}
El \textit{fuzzy hashing}, o \textit{hashing} difuso, es una técnica que extiende el concepto del \textit{hashing} tradicional al permitir la comparación de archivos basada en similitudes parciales en lugar de una coincidencia exacta. A diferencia del \textit{hashing} convencional, que opera sobre el archivo completo, el \textit{fuzzy hashing} divide el archivo en bloques o segmentos y calcula un \textit{hash} para cada uno de ellos. Este enfoque por bloques permite identificar similitudes entre archivos incluso cuando solo una porción de su contenido ha sido modificada.

El \textit{fuzzy hashing} es particularmente útil en el análisis forense digital y la detección de \textit{malware}, ya que permite identificar variantes de archivos maliciosos que han sido modificaods para evadir su detección, pero que conservan partes significativas de su código original. Al comparar dos \textit{hashes} difusos, es posible calcular un grado de similitud basado en la cantidad de bloques que coinciden entre ambos. Esto se logra mediante algoritmos especializados como SSDeep o TLSH, que están diseñados para generar \textit{hashes} difusos y medir la similitud entre ellos.

\section{Conceptos de LaTeX}
En aquellos proyectos que necesiten para su comprensión y desarrollo de unos conceptos teóricos de una determinada materia o de un determinado dominio de conocimiento, debe existir un apartado que sintetice dichos conceptos.

Algunos conceptos teóricos de \LaTeX{} \footnote{Créditos a los proyectos de Álvaro López Cantero: Configurador de Presupuestos y Roberto Izquierdo Amo: PLQuiz}.

\section{Secciones}

Las secciones se incluyen con el comando section.

\subsection{Subsecciones}

Además de secciones tenemos subsecciones.

\subsubsection{Subsubsecciones}

Y subsecciones. 


\section{Referencias}

Las referencias se incluyen en el texto usando cite~\cite{wiki:latex}. Para citar webs, artículos o libros~\cite{koza92}, si se desean citar más de uno en el mismo lugar~\cite{bortolot2005, koza92}.


\section{Imágenes}

Se pueden incluir imágenes con los comandos standard de \LaTeX, pero esta plantilla dispone de comandos propios como por ejemplo el siguiente:

\imagen{escudoInfor}{Autómata para una expresión vacía}{.5}



\section{Listas de items}

Existen tres posibilidades:

\begin{itemize}
	\item primer item.
	\item segundo item.
\end{itemize}

\begin{enumerate}
	\item primer item.
	\item segundo item.
\end{enumerate}

\begin{description}
	\item[Primer item] más información sobre el primer item.
	\item[Segundo item] más información sobre el segundo item.
\end{description}
	
\begin{itemize}
\item 
\end{itemize}

\section{Tablas}

Igualmente se pueden usar los comandos específicos de \LaTeX o bien usar alguno de los comandos de la plantilla.

\tablaSmall{Herramientas y tecnologías utilizadas en cada parte del proyecto}{l c c c c}{herramientasportipodeuso}
{ \multicolumn{1}{l}{Herramientas} & App AngularJS & API REST & BD & Memoria \\}{ 
HTML5 & X & & &\\
CSS3 & X & & &\\
BOOTSTRAP & X & & &\\
JavaScript & X & & &\\
AngularJS & X & & &\\
Bower & X & & &\\
PHP & & X & &\\
Karma + Jasmine & X & & &\\
Slim framework & & X & &\\
Idiorm & & X & &\\
Composer & & X & &\\
JSON & X & X & &\\
PhpStorm & X & X & &\\
MySQL & & & X &\\
PhpMyAdmin & & & X &\\
Git + BitBucket & X & X & X & X\\
Mik\TeX{} & & & & X\\
\TeX{}Maker & & & & X\\
Astah & & & & X\\
Balsamiq Mockups & X & & &\\
VersionOne & X & X & X & X\\
} 
