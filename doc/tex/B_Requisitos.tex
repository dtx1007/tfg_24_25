\apendice{Especificación de Requisitos}

\section{Introducción}

Este apéndice documenta la especificación de requisitos del trabajo desarrollado. Su propósito es definir de manera formal y detallada las capacidades, características y restricciones del sistema. Se establecen los objetivos generales del proyecto, se desglosa un catálogo de requisitos funcionales y no funcionales, y finalmente, se describen los principales casos de uso que ilustran la interacción del usuario con la aplicación.

\section{Objetivos generales}
Los objetivos generales representan las metas de alto nivel que se buscaron alcanzar con la realización de este trabajo, combinando tanto las aspiraciones de investigación como los entregables prácticos.

\begin{enumerate}
	\item \textbf{Investigar el estado del arte:} Realizar un análisis de la literatura científica para comprender las técnicas actuales de detección de \textit{malware} con IA y posicionar el proyecto en el panorama actual.
	
	\item \textbf{Desarrollar un sistema de detección de extremo a extremo:} Construir un \textit{pipeline} completo, desde la recolección de datos y la extracción de características hasta el entrenamiento y la evaluación de un modelo funcional.
	
	\item \textbf{Alcanzar un alto rendimiento predictivo:} Lograr que los modelos desarrollados obtengan métricas de clasificación altas, con un enfoque especial maximizar el \textit{recall} para minimizar los falsos negativos.
	
	\item \textbf{Comparar diferentes arquitecturas de modelos:} Evaluar y contrastar el rendimiento de una red neuronal profunda frente a algoritmos de aprendizaje automático clásicos para obtener conclusiones sobre la eficacia de cada enfoque.
	
	\item \textbf{Garantizar la interpretabilidad del sistema:} Implementar técnicas que permitan explicar las decisiones de los modelos, aportando transparencia y confianza a los resultados.
	
	\item \textbf{Crear una aplicación de demostración:} Desarrollar una interfaz web interactiva que permita a un usuario probar y visualizar el funcionamiento de todo el sistema.
\end{enumerate}

\section{Catálogo de requisitos}
A continuación se presenta el catálogo detallado de requisitos que el sistema debe satisfacer.

\subsection{Requisitos funcionales}
Los requisitos funcionales (RF) especifican lo que el sistema debe hacer. Describen las funcionalidades, tareas y servicios que la aplicación final debe proporcionar al usuario para cumplir con su propósito.

\begin{itemize}
	\item \textbf{RF-1: Carga de archivos APK.} El sistema debe proporcionar una interfaz que permita al usuario seleccionar y subir un archivo con formato \texttt{.apk} para su posterior análisis.
	
	\item \textbf{RF-2: Análisis y clasificación de la aplicación.} Una vez subida una APK, el sistema debe ejecutar el \textit{pipeline} de análisis completo. Esto incluye la extracción de características estáticas, el preprocesamiento de los datos a través del \textit{embedder} y la ejecución de la inferencia con todos los modelos entrenados (la red neuronal y los clasificadores clásicos).
	
	\item \textbf{RF-3: Visualización de las predicciones.} La aplicación debe presentar al usuario los resultados de la clasificación de forma clara. Debe mostrar un veredicto general y una tabla detallada con la predicción de cada modelo individual, incluyendo los porcentajes de confianza para las clases <<benigno>> y <<malicioso>>.
	
	\begin{itemize}
		\item \textbf{RF-3.1: Transparencia del proceso de extracción.} Para que el proceso no sea una <<caja negra>>, la interfaz debe permitir al usuario inspeccionar las características <<brutas>> que han sido extraídas de la APK. Esto incluye las diferentes listas de permisos, actividades, servicios, receptores, así como propiedades del archivo como su tamaño o su \textit{fuzzy hash}.
		
		\item \textbf{RF-3.2: Visualización de datos procesados.} Además de los datos en <<bruto>>, el sistema debe mostrar la representación numérica en la que se transforman. Esto implica visualizar tanto los datos \textit{tokenizados} y escalados como el vector de \textit{embeddings} final que se introduce en los clasificadores.
	\end{itemize}
	
	\item \textbf{RF-4: Interpretabilidad de las predicciones.} La aplicación debe ofrecer explicaciones sobre las decisiones del modelo. Para ello, deberá generar y mostrar un conjunto de gráficos de SHAP, incluyendo la importancia global de las características y análisis locales que detallen qué factores han influido en la predicción de la muestra actual.
	
	\item \textbf{RF-5: Visualización del espacio de características.} El sistema debe ser capaz de generar una proyección 2D del espacio de \textit{embeddings} mediante UMAP. En esta visualización, se debe mostrar la distribución de las muestras del \textit{dataset} de fondo y resaltar la posición de la APK recién analizada, permitiendo al usuario entender su ubicación relativa respecto a las clases conocidas.
	
	\item \textbf{RF-6: Gestión del historial de análisis.} El sistema debe mantener un historial de las aplicaciones analizadas durante la sesión activa del usuario. La interfaz debe permitir al usuario seleccionar una entrada del historial para volver a cargar sus resultados y ofrecer una opción para borrar todo el historial de la sesión.
\end{itemize}

\subsection{Requisitos no funcionales}
Los requisitos no funcionales (RNF) describen los atributos de calidad y las restricciones bajo las cuales el sistema debe operar. No se refieren a qué hace el sistema, sino a <<cómo>> lo hace, definiendo aspectos como su rendimiento, fiabilidad o portabilidad.

\begin{itemize}
	\item \textbf{RNF-1: Rendimiento y fiabilidad de los modelos.} Los distintos clasificadores deben alcanzar un alto nivel de rendimiento, definido por métricas de evaluación estándar. Específicamente, se establece como requisito clave obtener una métrica de \textit{recall} cercana al 98\%, garantizando una detección muy alta de las muestras maliciosas.
	
	\item \textbf{RNF-2: Eficiencia del análisis.} El tiempo total desde que el usuario sube una APK hasta que recibe un resultado completo (incluyendo la extracción, el preprocesamiento y la inferencia) debe ser razonable para una buena experiencia de usuario, idealmente completándose en menos de un par de minutos.
	
	\item \textbf{RNF-3: Portabilidad y facilidad de despliegue.} Todo el sistema, incluyendo la aplicación web, el modelo y sus dependencias, debe estar empaquetado en un contenedor Docker. Esto asegura que la aplicación sea portable y pueda ser desplegada de forma sencilla y consistente en diferentes sistemas operativos.
	
	\item \textbf{RNF-4: Usabilidad de la interfaz.} La aplicación web debe tener una interfaz de usuario clara, intuitiva y fácil de navegar para un perfil de usuario con conocimientos técnicos, pero no necesariamente experto en inteligencia artificial. La información debe presentarse de forma organizada y comprensible.
	
	\item \textbf{RNF-5: Modularidad del código.} La base de código del proyecto debe seguir un diseño modular que separe claramente las distintas responsabilidades (extracción de datos, arquitectura del modelo, entrenamiento, etc.). Esto facilita el mantenimiento, la experimentación y el posible desarrollo futuro del sistema.
\end{itemize}

\section{Casos de uso}
Los casos de uso describen las interacciones entre un actor (en este caso, el <<Usuario/Analista>>) y el sistema para alcanzar un objetivo. A continuación se presenta un diagrama esquemático (Figura \ref{fig:use_cases_diagram}) y se detallan los tres casos de uso principales.

% Diagrama Esquelético de Casos de Uso:
% Actor: Usuario/Analista
% Casos de Uso:
% 1. (Usuario/Analista) -> Analizar una aplicación APK
% 2. (Usuario/Analista) -> Interpretar resultados de una predicción
% 3. (Usuario/Analista) -> Gestionar historial de análisis

\begin{table}[H]
	\centering
	\begin{tabularx}{\linewidth}{ p{0.21\columnwidth} p{0.71\columnwidth} }
		\toprule
		\textbf{CU-1}    & \textbf{Analizar una nueva aplicación APK}\\
		\toprule
		\textbf{Versión}              & 1.0    \\
		\textbf{Autor}                & David Cezar Toderas \\
		\textbf{Requisitos asociados} & RF-1, RF-2, RF-3 \\
		\textbf{Descripción}          & El usuario sube un archivo APK al sistema para que este sea analizado por los modelos de IA y se muestren los resultados de la clasificación. \\
		\textbf{Precondición}         & El usuario tiene un archivo \texttt{.apk} válido y la aplicación web está en ejecución. \\
		\textbf{Acciones}             &
		\begin{enumerate}
			\def\labelenumi{\arabic{enumi}.}
			\tightlist
			\item El usuario accede a la aplicación web.
			\item El usuario arrastra y suelta un archivo APK en la zona de carga o lo selecciona mediante el explorador de archivos.
			\item El usuario presiona el botón <<Analizar APK>>.
			\item El sistema procesa el archivo: extrae las características, las pasa por el \textit{embedder} y realiza la inferencia con todos los modelos.
			\item La interfaz se actualiza para mostrar la pestaña de <<Predicciones>> con el veredicto de cada modelo.
			\item El análisis se añade al historial de la sesión.
		\end{enumerate}\\
		\textbf{Postcondición}        & Se muestra una predicción de clasificación para la APK subida. \\
		\textbf{Excepciones}          & \textbf{E-1.1:} Si el archivo subido no es un APK válido, el sistema muestra un mensaje de error. \newline \textbf{E-1.2:} Si el archivo supera el límite de tamaño, el sistema muestra una advertencia y no permite el análisis. \\
		\textbf{Importancia}          & Alta \\
		\bottomrule
	\end{tabularx}
	\caption{CU-1 Analizar una nueva aplicación APK.}
\end{table}


\begin{table}[H]
	\centering
	\begin{tabularx}{\linewidth}{ p{0.21\columnwidth} p{0.71\columnwidth} }
		\toprule
		\textbf{CU-2}    & \textbf{Interpretar el resultado de un análisis}\\
		\toprule
		\textbf{Versión}              & 1.0    \\
		\textbf{Autor}                & David Cezar Toderas \\
		\textbf{Requisitos asociados} & RF-3.1, RF-3.2, RF-4, RF-5 \\
		\textbf{Descripción}          & Tras analizar un APK, el usuario explora las diferentes pestañas de la interfaz para comprender en profundidad el resultado y el razonamiento del modelo. \\
		\textbf{Precondición}         & Se ha completado con éxito un análisis (CU-1). \\
		\textbf{Acciones}             &
		\begin{enumerate}
			\def\labelenumi{\arabic{enumi}.}
			\tightlist
			\item El usuario navega a la pestaña <<Características Extraídas>> para ver los datos brutos y procesados.
			\item El usuario navega a la pestaña <<Explicaciones>> para visualizar los gráficos de importancia de características de SHAP.
			\item El usuario navega a la pestaña <<UMAP>> para ver la proyección del \textit{embedding} de la aplicación analizada.
		\end{enumerate}\\
		\textbf{Postcondición}        & El usuario obtiene una visión detallada de los datos y las justificaciones que respaldan la predicción del modelo. \\
		\textbf{Excepciones}          & Ninguna. \\
		\textbf{Importancia}          & Alta \\
		\bottomrule
	\end{tabularx}
	\caption{CU-2 Interpretar el resultado de un análisis.}
\end{table}

\begin{table}[H]
	\centering
	\begin{tabularx}{\linewidth}{ p{0.21\columnwidth} p{0.71\columnwidth} }
		\toprule
		\textbf{CU-3}    & \textbf{Gestionar el historial de análisis}\\
		\toprule
		\textbf{Versión}              & 1.0    \\
		\textbf{Autor}                & David Cezar Toderas \\
		\textbf{Requisitos asociados} & RF-6 \\
		\textbf{Descripción}          & El usuario interactúa con el historial de análisis de la sesión actual para consultar resultados anteriores o para limpiar la lista. \\
		\textbf{Precondición}         & Se ha analizado al menos una aplicación durante la sesión actual. \\
		\textbf{Acciones}             &
		\textbf{Flujo A: Consultar un análisis anterior}
		\begin{enumerate}
			\def\labelenumi{\arabic{enumi}.}
			\tightlist
			\item El usuario hace clic sobre el nombre de un archivo en la lista del <<Historial de Análisis>>.
			\item El sistema carga en la vista principal todos los datos y resultados correspondientes a ese análisis.
		\end{enumerate}
		\textbf{Flujo B: Limpiar el historial}
		\begin{enumerate}
			\def\labelenumi{\arabic{enumi}.}
			\tightlist
			\item El usuario hace clic en el botón <<Limpiar Historial>>.
			\item El sistema borra todas las entradas del historial de la sesión.
		\end{enumerate}\\
		\textbf{Postcondición}        & \textbf{Flujo A:} La interfaz muestra los resultados del análisis seleccionado. \newline \textbf{Flujo B:} La lista del historial de análisis queda vacía. \\
		\textbf{Excepciones}          & Ninguna. \\
		\textbf{Importancia}          & Media \\
		\bottomrule
	\end{tabularx}
	\caption{CU-3 Gestionar el historial de análisis.}
\end{table}

\imagen{use_cases_diagram}{Diagrama de casos de uso de la aplicación de análisis de \textit{malware}.}
